\documentclass[twocolumn]{article}

% Language setting
% Replace `english' with e.g. `spanish' to change the document language
\usepackage[english]{babel}

% Set page size and margins
% Replace `letterpaper' with`a4paper' for UK/EU standard size
\usepackage[letterpaper,top=2cm,bottom=2cm,left=3cm,right=3cm,marginparwidth=1.75cm]{geometry}

% Useful packages
\usepackage{amsmath}
\usepackage{graphicx}
\usepackage[colorlinks=true, allcolors=blue]{hyperref}
\usepackage{outlines}

\title{Title - Keywords + Reference to application/Context}
\author{Authors}

\begin{document}
\maketitle

\begin{abstract}
\begin{outline}
    \1 Context [1 to 2 ph]
    \2 Theme/ Broad intro
    \2 Identified gap in research
    \2 Problem statement
    \1 Contribution
    \2 \emph{This work presents ...}
    \1 Related work
    \1 Conclusion on results \emph{Contrary to previous solutions, the method introduced in this work closes the gap}
\end{outline}

\end{abstract}

\section{Introduction}
\begin{outline}
\1 Broad Introduction
\1 Context
\1 Application Examples [Understandable by reviewers outside of the field]
\2 Inspection/ Maintenance of offshore structures/systems
\2 Biodiversity monitoring
\2 Explore sunk wreckage
\2 Help understanding of the ocean
\2 Unlock new technology (offshore energy production)
\2 Replace arduous/dangerous work
\2 Ease work conditions
\2 ...
\1 Problem Statement [Make things very clear]
\2 Pas de GPS sous l'eau
\2 Pas de methode de localisation fiable
\2 Manque de controlleur robuste et simple sans modele
\2 Need for efficient obstacle avoidance solution for sailboat
\2 Limits due to ROV cables
\2 Limited autonomy needs to be enhanced
\2 Limited maneuverability needs to be enhanced
\2 Limitations in computational power
\2 Limitations in onboard volume/space/weight
\2 Aim toward frugality
\2 Need for new docking solutions
\2 Need for new locomotion solution
\2 ...
\end{outline}

\subsection{Related Works}
mettre en avant les problemes
\begin{outline}
\1 Ancestral methods in the field
\2 Methods to compare to (PID/SMC/LQR/Kalman)
\1 Recent works in the field (less than 5/10 years)
\2 Methods to compare to (MPC/ADRC/IA/NN)
\2 Pick 2 to 4 methods total to compare to
\2 Can be some of your previous works
\1 Broadly related works [For reviewers outside of the field]
\2 Different domain but same goal
\3 Terrestrial/Flying
\3 Biomimetism
\3 Manipulator arms
\3 Network management
\3 ...
\2 Different application but comparable problems
\end{outline}

\subsection{Contribution}
bullet points
\begin{outline}
\1 Description of the concept/ Method / Approach / Hypothess
\1 Bullet points contributions
\1 Results obtained
\1 Why/How is it novelty ? 
\2 \emph{Closes the gap in research}
\2 \emph{Never done before}
\2 \emph{Contrary to \emph{[methods introduced in related work]} this new approach allows ...}
\end{outline}

\subsection{Outline}

\section{Problem Definition and Preliminaries [Material and Methods]}
\subsection{Problem Definition}
\begin{outline}
\1 If possible : relate an application problem to math constraint (e.g. gimbal loc)
\1 Detail application
\2 Trajectories
\2 Specific need
\2 Specific Environment
\1 Usual assumptions + why [Bullet points]
\2 Horizontal plane \emph{aucun effet des phénomènes hors du plan horizontal sur la solution}
\2 Constant/slowly varying disturbance
\2 Bounded disturbance
\2 Bounded inputs
\2 Recall limitations from intro
\end{outline}

\subsection{Preliminaries}
\begin{outline}
\1 Mathematical tools 
\2 Interval analysis
\2 Command methods
\2 Optimization problems ...
\1 Techno
\2 Description of the system
\2 Sensors
\2 Computational power / Architecture
\1 Model description
\1 Notation
\end{outline}

\section{Main contributions}
\subsection{Complicated/Unfamiliar Hypotheses demonstration}
\begin{outline}
\1 If some assumptions are not usual: demonstrate they are well thought
\end{outline}
\subsection{Design of the solution}
\begin{outline}
\1 Equations
\1 Technological solution
\1 Algorithm
\1 ...
\end{outline}
\subsection{Analysis}
\begin{outline}
\1 Math : Theorem/Proof
\2 Theorem : \emph{Controller [] introduced in [] ensures stability of system [] towards reference []}
\2 Proof : \emph{Lyapunov analysis}
\1 ?
\end{outline}

\section{Results}
\subsection{Simulation}
\begin{outline}
\1 Simulator
\2 Assumptions
\2 Modelled phenomena (current, waves, wind, ...)
\2 Type of simulator 
\3 Python/Matlab : Math results (control/stability...)
\3 Multiphysics/Gazebo : Application results (Obstacle avoidance, Robotic concept, ...)

\1 Comparative methods

\1 Results
\2 Concept alone + Discussion
\2 Comparative figure + Discussion
\2 Table of Results
\3 Comparison metrics

\end{outline}


\subsection{Experiments}
\begin{outline}
\1 Context of the experiment
\2 Consequences on the system


\1 Comparative methods

\1 Results
\2 Concept alone + Discussion
\2 Comparative figure + Discussion
\2 Table of Results
\3 Comparison metrics
\end{outline}

\section{Discussion / Conclusion}
\begin{outline}
\1 Go back to abstract/Intro
\1 Show how the gap in research has been closed
\2 Recall some comparative metrics
\2 Recall some outstanding performance
\1 Future works
\2 Future needs highlighted by your work
\2 Future steps to application of your work
\end{outline}


\bibliographystyle{alpha}
\bibliography{sample}

\end{document}